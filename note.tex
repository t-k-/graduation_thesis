\documentclass[]{article}
\usepackage{xeCJK}
\setCJKmainfont{WenQuanYi Zen Hei Sharp}
\begin{document}

\begin{titlepage}
\begin{center}
\LARGE 毕业设计理论笔记\\
[1.5cm]
\large{钟\ 威}
\vfill
\large 最后更改时间:\today \\
\end{center}
\end{titlepage}

\section{凸规划问题基础}

\subsection{凸集} 
($P_{17}$ 定义 1.2.1) 
\quad 
称S是凸集,如果对于任意的$x_1,x_2 \in S$和任意的$\lambda \in [0,1]$,都有:
$$ 
{\lambda}x_1 + (1 - {\lambda})x_2 \in S 
\eqno{(1.2.1)} $$

\subsection{凸函数}
($P_{17}$ 定义 1.2.3)
\quad 
任意的$ x_1,x_2 \in S $和任意的$ \lambda \in (0,1)$,都有:
$$ 
f({\lambda}x_1 + (1 - \lambda)x_2) \le {\lambda}f(x_1) + (1 - \lambda)f(x_2) 
\eqno{(1.2.2)} $$

\subsection{凸函数的充要条件}
($P_{19}$ 定理 1.2.4)
\quad 
$f(x)$是凸函数的充要条件是:对于S中的任意一点$\bar{x}$,都有:
$$ 
f(x) \ge f(\bar{x}) + \nabla f(\bar{x})^T(x-\bar{x})
\eqno{(1.2.6)} $$

\subsection{凸规划问题}
($P_{20}$ 定义 1.2.6)
$$
\min \qquad f_0(x), \ x \in R^n 
\eqno{(1.2.10)} $$
$$
s.t. \qquad f_i(x) \le 0, \ i = 1, \ldots , m 
\eqno{(1.2.11)} $$

\begin{eqnarray}
\label{1.2.3}
asd&f = asf \\
\label{1.2.3}
fwer&gge = adf
\end{eqnarray}

\end{document}
